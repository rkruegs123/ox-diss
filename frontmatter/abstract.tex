\begin{abstract}

Traditional quantum circuit optimization is performed directly at the circuit level.
Alternatively, a quantum circuit can be translated to a ZX-diagram which can be simplified using the rules of the ZX-calculus, after which a simplified circuit can be extracted.
However, the best-known extraction procedures can drastically increase the number of 2-qubit gates.
In this work, we take advantage of the fact that local changes in a ZX-diagram can drastically affect the complexity of the extracted circuit.
We use a pair of congruences (i.e., non-simplification rewrite rules) based on the graph-theoretic notions of local complementation and pivoting to generate local variants of a simplified ZX-diagram.
We explore the space of equivalent ZX-diagrams generated by these congruences using simulated annealing and genetic algorithms to obtain a simplified circuit with fewer 2-qubit gates.
On randomly generated circuits, our method can reliably outperform state-of-the-art optimization techniques for low-qubit ($<10$) circuits.
On a set of previously reported benchmark circuits with $\leq 14$ qubits, our method outperforms off-the-shelf methods in 87\% of cases, consistently reducing overall circuit complexity by an additional \textasciitilde 15-30\% and eliminating up to 46\% of 2-qubit gates.
% On a set of previously reported benchmark circuits, our method can reduce the number of 2-qubit gates by up to 46\% and can consistently reduce overall circuit complexity by \textasciitilde 20-30\% compared to single-digit reductions offered by off-the-shelf methods.
% and serves as a proof-of-concept for a new circuit optimization strategy in the ZX-calculus.

\end{abstract}
