\chapter[Methods]{Methods} \label{ch:methods}

At a high level, the goal of any quantum circuit optimization is to reduce the complexity of the input circuit.
A core assumption of circuit optimization in the ZX-calculus is that ZX-diagrams with fewer spiders typically correspond to simpler circuits.
Earlier, however, we noted that local changes (e.g., connectivity) in a ZX-diagram can drastically affect the complexity of the associated circuit obtained via extraction.
However, no existing optimization procedure over ZX-diagrams addresses this and searches this local space.
The \codeword{full_reduce} method described in Section \ref{sec:zx-circ-opt} is the quintessential example of this lost opportunity.
After the two simplification rules can no longer be applied, the final circuit is taken to be that which is extracted from the resulting simplified ZX-diagram;
however, there may be an equivalent ZX-diagram lurking nearby whose associated circuit is markedly less complex.

In this thesis, we explore the utility of searching this local space of equivalent ZX-diagrams.
To do so, we first need rewrite rules that modify a ZX-diagram in a manner other than spider removal that may reduce circuit complexity.
Indeed, a rule that introduces several additional spiders alongside changes in connectivity may in turn yield a less complex circuit.
We refer to these rewrite rules as \emph{congruences}.
Given a set of congruences, we then need a procedure to search the space of equivalent ZX-diagrams generated by an input ZX-diagram and these rules.
We can then devise strategies for incorporating this local search into existing methods for circuit optimization in the ZX-calculus.

We present the methods of our work in this order.
First, we generalize the original (non-simplification) variants of local complementation and pivoting to ZX-diagrams with arbitrary phases.
We then describe two search procedures, simulated annealing (SA) and genetic algorithms (GA), for searching the space of ZX-diagrams generated by these congruences.
We also define a measure of circuit complexity and discuss candidate objective functions to guide search. % to the ZX-diagram whose circuit has minimal complexity.
Lastly, we discuss how this search can be incorporated into existing optimization pipelines.
Our primary strategy is to first simplify a ZX-diagram using existing methods and subsequently search the local variants of the simplified ZX-diagram.
Alternatively, we can use SA or GA as the principal means of optimization.
In this more ambitious approach, congruences can be combined with simplification rules to form an action set over which search is applied.

\section{Congruences}

The non-simplification versions of local complementation and pivoting presented in Section \ref{sec:zx-circ-opt} embody the desired properties of congruences.
They change the connectivity of the ZX-diagram without introducing an unwieldy number of additional gates, presenting an opportunity for a potential reduction in circuit complexity.
However, Equations \ref{eq:gs-local-comp} and \ref{eq:gs-pivot} only apply to spiders with zero phase and a single wire.

% Useful for spacing:
% https://tex.stackexchange.com/questions/54587/vertical-spacing-within-align-environment-accounting-for-fractions
We can apply the rules of the ZX-calculus and Equation \ref{eq:gs-local-comp} to generalize local complementation to arbitrary phases ($\alpha_i, \beta_i \in [0, 2 \pi)$):
% {\allowdisplaybreaks
\begin{spreadlines}{0.8em}% tweak
  \begin{align*}
    \tikzfig{gen-lc-single/0} &\stackrel{(\bm f)}{=} \tikzfig{gen-lc-single/1} \\
    &\stackrel{(\ref{eq:gs-local-comp})}{=} \tikzfig{gen-lc-single/2} \\
    &\stackrel{(\bm i1)}{=} \tikzfig{gen-lc-single/3} \\
    &\stackrel{(\bm f)}{=} \tikzfig{gen-lc-single/4} \\
    &\stackrel{(\bm f)}{=} \tikzfig{gen-lc-single/5} \\[0.8em]
    &\stackrel{(\ref{eq:had-short})}{=} \tikzfig{gen-lc-single/6}\stepcounter{equation}\tag{\theequation}\label{eq:gen-phase-lc}
  \end{align*}
\end{spreadlines}
% }
Equation \ref{eq:gen-phase-lc} can be easily extended to apply for an arbitrary number of wires connected to each spider:
% {\allowdisplaybreaks
\begin{spreadlines}{0.8em}% tweak
  \begin{align*}
    \tikzfig{gen-lc-mul/0} &\stackrel{(\bm f)}{=} \tikzfig{gen-lc-mul/1} \\
    &\stackrel{(\ref{eq:gen-phase-lc})}{=} \tikzfig{gen-lc-mul/2} \\
    &\stackrel{(\bm f)}{=} \tikzfig{gen-lc-mul/3}\stepcounter{equation}\tag{\theequation}\label{eq:gen-io-lc}
  \end{align*}
\end{spreadlines}
% }
% Equation \ref{eq:gs-pivot}:
Similarly, we can generalize pivoting to arbitrary phases:
% {\allowdisplaybreaks
\begin{spreadlines}{0.8em}% tweak
  \begin{align*}
    \tikzfig{gen-pivot-single/0} &\stackrel{(\bm f)}{=} \tikzfig{gen-pivot-single/1} \\
    &\stackrel{(\ref{eq:gs-pivot})}{=} \tikzfig{gen-pivot-single/2} \\
    &\stackrel{(\bm f)}{=} \tikzfig{gen-pivot-single/3}\stepcounter{equation}\tag{\theequation}\label{eq:gen-phase-pivot}
  \end{align*}
\end{spreadlines}
% }
Again, we can extend this rewrite rule for arbitrary wiring:
\begin{spreadlines}{0.8em}% tweak
  \begin{align*}
    \tikzfig{gen-pivot-mul/0} &\stackrel{(\bm f)}{=} \tikzfig{gen-pivot-mul/1} \\
    &\stackrel{(\ref{eq:gen-phase-pivot})}{=} \tikzfig{gen-pivot-mul/2} \\
    &\stackrel{(\bm f)}{=} \tikzfig{gen-pivot-mul/3}\stepcounter{equation}\tag{\theequation}\label{eq:gen-io-pivot}
  \end{align*}
\end{spreadlines}

TODO: Short summary paragraph of congruences. Maybe something like rules X and Y will be our congruences for generating ZX-diagrams with similar complexity whose circuits may have different complexities.

\section{Search Procedures}

TODO: Here we describe some search procedures to apply the congruences

\subsection{Simulated Annealing}

TODO

\subsection{Genetic Algorithms}

TODO

\subsection{Target Functions}

TODO: weighted 2-qubit gates, number of edges, etc

% look up ion trap papers, single qubit gates with 99.9 fidelity and two qubit with 99 fidelity. 10 perc difference in fidelity. chris ballance

\section{Circuit Optimization Strategies}

TODO: Here we describe how we use conruences and these search procedures for quantum circuit optimization. First describe primary method: searching local space after simplification. Then describe several other variants.

FIXME: Other, more general combinations (e.g., GA with all simps, or search with other ``safe'' procedures). Maybe simplifying to fixpoint isn't optimal, and introduces too many edges that lead to CNOTs to get us marginal T-count gain.

% subsection: after ZX-diagram reduction. try different reductions. Post-Simplification of ZX-Diagram

% subsection: from scratch. just include congruences as a possible action along with all the others.
