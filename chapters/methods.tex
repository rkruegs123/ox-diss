\chapter[Methods]{Methods} \label{ch:methods}

TODO: Previous approaches failed to do FIXME. We want to do that. Here we describe methods for doing so.
TODO: First, we want rewrite rules that don't necessarily simplify but preserve focused gFlow and may change circuit extraction. Pivoting and LC are good for this because they change connectivity. Lets show that.
TODO: We then need a way to find the best application of these congruences to simplify circuit. We describe some general purpose search procedures that we use.
TODO: We then describe how we are going to use all of this for circuit optimization.

\section{Congruences}

FIXME: Congruences. We desire rewrite rules that won't make the graph about as complex but could lead to different extraction.

TODO: Proof in appendix

\section{Search Procedures}

TODO: Here we describe some search procedures to apply the congruences

\subsection{Simulated Annealing}

TODO

\subsection{Genetic Algorithms}

TODO

\subsection{Target Functions}

TODO: weighted 2-qubit gates, number of edges, etc

% look up ion trap papers, single qubit gates with 99.9 fidelity and two qubit with 99 fidelity. 10 perc difference in fidelity. chris ballance

\section{Circuit Optimization Strategies}

TODO: Here we describe how we use conruences and these search procedures for quantum circuit optimization. First describe primary method: searching local space after simplification. Then describe several other variants.

FIXME: Other, more general combinations (e.g., GA with all simps, or search with other ``safe'' procedures)

% subsection: after ZX-diagram reduction. try different reductions. Post-Simplification of ZX-Diagram

% subsection: from scratch. just include congruences as a possible action along with all the others.
