%
% File:   test1.qasm
% Date:   22-Mar-04
% Author: I. Chuang <ichuang@mit.edu>
%
% Sample qasm input file - EPR creation
%
%         qubit 	q0
%         qubit 	q1
%         qubit 	q2
% 
% 	S	q0
%         cnot	q0,q1
%         h       q1
%         cnot    q1,q2

%  Time 01:
%    Gate 00 S(q0)
%  Time 02:
%    Gate 01 cnot(q0,q1)
%  Time 03:
%    Gate 02 h(q1)
%  Time 04:
%    Gate 03 cnot(q1,q2)

% Qubit circuit matrix:
%
% q0: gAxA, gBxA, n  , n  , n  
% q1: n  , gBxB, gCxB, gDxB, n  
% q2: n  , n  , n  , gDxC, n  

\documentclass[11pt]{article}
\input{xyqcirc.tex}

% definitions for the circuit elements

\def\gAxA{\op{S}\w\A{gAxA}}
\def\gBxA{\b\w\A{gBxA}}
\def\gBxB{\o\w\A{gBxB}}
\def\gCxB{\op{H}\w\A{gCxB}}
\def\gDxB{\b\w\A{gDxB}}
\def\gDxC{\o\w\A{gDxC}}

% definitions for bit labels and initial states

\def\bA{ \q{q_{0}}}
\def\bB{ \q{q_{1}}}
\def\bC{ \q{q_{2}}}

% The quantum circuit as an xymatrix

\xymatrix@R=5pt@C=10pt{
    \bA & \gAxA &\gBxA &\n   &\n   &\n  
\\  \bB & \n   &\gBxB &\gCxB &\gDxB &\n  
\\  \bC & \n   &\n   &\n   &\gDxC &\n  
%
% Vertical lines and other post-xymatrix latex
%
\ar@{-}"gBxB";"gBxA"
\ar@{-}"gDxC";"gDxB"
}

\end{document}
